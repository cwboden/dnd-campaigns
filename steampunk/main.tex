% Changing book to article will make the footers match on each page,
% rather than alternate every other.
%
% Note that the article class does not have chapters.
\documentclass[letterpaper,10pt,twoside,twocolumn,openany]{dndbook}

% Use babel or polyglossia to automatically redefine macros for terms
% Armor Class, Level, etc...
% Default output is in English; captions are located in lib/dndstring-captions.sty.
% If no captions exist for a language, English will be used.
%1. To load a language with babel:
%	\usepackage[<lang>]{babel}
%2. To load a language with polyglossia:
%	\usepackage{polyglossia}
%	\setdefaultlanguage{<lang>}
\usepackage[english]{babel}
%usepackage[italian]{babel}
% For further options (multilanguage documents, hypenations, language environments...)
% please refer to babel/polyglossia's documentation.

\usepackage[utf8]{inputenc}
\usepackage{hang}
\usepackage{lipsum}
\usepackage{listings}

\lstset{%
  basicstyle=\ttfamily,
  language=[LaTeX]{TeX},
}

% Start document
\begin{document}

% Your content goes here

% Comment this out if you're using the article class.
\chapter{Chapter 1: The Train to Steelport}

\section{Overview}

The first part of the adventure covers the adventurers on their way to Steelport.  They are looking for an unknown treasure that is rumored to give a wish to anyone who can find it.  There is a tip that someone aboard the train knows something about this treasure.  

The players will each receive some determining factor hinting at who the person of interest is.  They can then decide how they want to proceed, whether to work together, work alone, try to get the person alone or not.  But if they delay too long or are unable to find their target quickly, they may find someone else is able to get to their target first.  

Feel free to read this to set the scene aboard the train as it departs from Neverwinter.
 
\begin{quotebox}
A steam whistle blows as the conductor signals that the train has boarded all its passengers. As the wheels screeched into motion, the passengers aboard jolted in their seats as the engine chugged to life. 
\end{quotebox}

\section{Inside the Boxcar}
The story begins with the adventurers sharing the same box car on the train. Ask them to introduce one another and perhaps engage in some small talk.  This can also be a time to describe the rest of the train.  

\begin{quotebox}
The scratched, faded wood of the train car's walls creak as the wheels rhythmically rattle on the track.  The trees of Neverwinter become sparse as the train distances itself from the rural city.
\end{quotebox}

The point is to make sure players are able to build the story on their own.  There's not a specific, predetermined story within the car, more freestyling based on how the players decide to approach their target, or what happens if they are too slow in doing so.

\section{Exploring the Train}
The train consists of multiple boxcars that the players can explore.  They first begin in the passengers' boxcar.  Furthermore, the train has multiple passengers that the players can interact with. The cars and passengers in each car, in order, are as follows:

\begin{dndtable}[l l l]
  \textbf{Train Car}  & \textbf{Passenger}  & \textbf{Identity} \\
  Locomotive          & Casey Jones         & Engineer \\
  Coal Car            & \textit{Nobody}     & \\
  Dining Car          & Dantonio Rivera     & Chef \\
  Passenger Car       & Charlene Francios   & Warlock  \\
                      & Cassandra Troy      & Cleric \\
                      & Drake Kahn          & Fence / PoI \\
  Lounge Car          & Bartholomew Higgins & High-Roller \\
                      & Helga Boneskull     & Wrestler \\
                      & Rasheed Jules       & Bard \\
  Luggage Car         & \textit{Nobody}     & \\
  Caboose             & Jenkins Malone      & Steward
\end{dndtable}

\subsection{Finding the Informant}
The players are all looking for Drake Kahn, but its important that they all know different facts about the character.  Here are some possible facts:

\begin{itemize}
  \item The informant is wearing an amethyst ring
  \item The informant is carrying a short sword
  \item The informant has a tattoo of a wolf
\end{itemize}

\subsection{Meeting Passengers}
Aboard the train are many other passengers that the adventurers can meet.  Below are some brief descriptions of each character.  Make sure to read them so that players know what to look for in their person of interest.  Some of the clues players have may require them to interact with the passengers; this is good!

\subsubsection{Casey Jones \textit{The Engineer}}
Casey Jones is the engineer running the train.  It is unlikely that the players will meet her face-to-face, since she is barricaded in the locomotive.  She is an extremely anxious dwarf who fears wide-open spaces. As a result, she stays boxed up in the locomotive while the train is moving.

\begin{quotebox}
As you approach the rear-end of the locomotive, a loudspeaker blares to life.  A female voice speaks, `W-what are you doing up here?  Please return to the passenger cabins'
\end{quotebox}

\subsubsection{Dantonio Rivera \textit{The Chef}}
This chef loves making improvisations in the kitchen.  He is an always-smiling, broad-shouldered human who is missing two fingers on his left hand from a cooking accident.  He is happy to make food to order for any of the passengers, despite the limited supply of food in the kitchen.

\begin{quotebox}
Entering the kitchen, you are met with a sizzling sound paired with delicious, hearty fragarance.  A bald-headed, broad-shouldered human man looks up at you from behind the kitchen counter and grins.  `What can I get you?' he asks, gesturing to you with his bare palm where you see two missing fingers.  His other hand, covered with a winding, snake tattoo, continues to stir-fry a pan on the stove.
\end{quotebox}

\subsubsection{Bartholomew Higgins \textit{The High-Roller}}
Higgins is a crude gnome travelling to Steelport to visit some of their high-end brothels.  He is a compulsive liar and surpsingly rich.  He feels that he is of higher social stature than anyone else on the train and thus will not treat them very nicely.

\begin{quotebox}
Sitting in the lounge, you see a slender, short gnome covered in a large, fur coat.  His hands are covered in rings of diamond, amethyst, garnet, emerald, and the like.  One of this hands grasps an amber-topped cane.  He briefly makes eye-contact with you as you scan the room, and he scowls and looks away quickly.
\end{quotebox}

\subsubsection{Helga Boneskull \textit{The Wrestler}}
Hailing from overseas, Helga is a half-orc travelling to Steelport as an up-and-coming wrestler.  She is entered in a wrestling tournament and believes she has a solid shot at winning. Despite her intimidating stature, she is incredibly sweet and will treat anyone with equal respect.  She has a soul-bound axe that can appear in her hands at any time.

\begin{quotebox}
Towering in the lounge is a female half-orc in a fashionable, crimson robe with white trim.  Her wrists are clad with bronze bracers that barely contain her bare, green arms.  Despite her intimidating stature, you watch her give a sharp-toothed grin, laughing with the half-elf sitting across from her.
\end{quotebox}

\subsubsection{Rasheed Jules \textit{The Bard}}
Rasheed is a half-elf bard whose nosy tendencies get him into trouble.  He is currently returning from a six-month trip in Neverwinter spent lying low from the mob bosses in Steelport.  He is extremely suspicious of anyone who attempts to get too friendly with him aboard the train.  A DC Perception check of DC 12 will clue players in that he is eyeing them suspiciously before they meet him.

\begin{quotebox}
Chatting with the half-orc in the lounge is a half-elf in dishevled performer's attire.  On his back is a lute, a shortbow, and a longsword.  He continues to crack jokes with the half-orc.
\end{quotebox}

\subsubsection{Cassandra Troy \textit{The Cleric}}
This cleric believes her constant nose bleeds indicate when an evil presence is active. This is false, but sometimes chance tends to work in her favor. Cassandra is a slender, sentimental halfling who is travelling to Steelport to put her skills to use in the city.

\begin{quotebox}
As you pass through the passenger car, you see a slender halfling seated alone in her section.  She stares pensively out the window, fiddling with a ring on her hand.  You can see her pack, open in the box, brimming with healing salves and bandages.
\end{quotebox}
 
\subsubsection{Drake Kahn \textit{The Person of Interest}}
Alcoholics are common, but functioning alcoholics are a different sort.  Kahn is a surprisingly respectful, mild-mannered dragonborn, even when drunk.  He used to be a fence for the Lobo gang of the Steelport mob and will trade knowledge for booze.  He is the target the players are seeking.

\begin{quotebox}
Passed out in one of the passenger boxes lies a dragonborn clad in leather armor.  He snores loudly, but you can see a short and longsword on his back as well has two amethyst rings on his left hand.
\end{quotebox}

\subsubsection{Charlene Francios \textit{The Warlock}}
Francios is a well-travelled, elderly warlock.  She has wispy, white hair, a slender frame, and a scratchy voice.  Despite her somewhat frail appearance and clueless demeanor, she is a powerful warlock who will defend herself if necessary.

\begin{quotebox}
An elderly woman sits peacefully in her section.  She is also asleep, but doesn't seem to move.  Her frail, left hand, resting on her billowing, blue robes, bears a single, golden ring.  A staff rests at her side.
\end{quotebox}

\subsubsection{Jenkins Malone \textit{The Steward}}
Jenkins is the steward of the Steelport train.  He will ask the players if their journey is going well and do his best to help solve any problems.  He is also a wizard who will try to stop any issues that arise on the plane.

\begin{quotebox}
`Hello, welcome to my office' says a tall, wispy, white-haired man, gesturing to the humble space within the caboose.  `Is your journey meeting your expectations? Or is there anything I can do to improve it?'  His rainbow bowtie glows gently in the dimly-lit room, drawing attention his ratty suit and cumberbund. 
\end{quotebox}

\section{The Murders}
Someone on the train is looking for the informant and aims to kill them.  They have been given the same clues as the players (except all 3), but they are indescriminantly killing those on the train that meet any of the descriptions.  As a result, the following passengers will be killed until the murderer is found.

\begin{itemize}
  \item Rasheed Jules
  \item Cassandra Troy
  \item Bartholomew Higgins
\end{itemize}

The murderer in question is Jenkins Malone.  \textit{(gasp!)}  The following sections will highlight how each of the individuals dies and what clues are left behind for the players.  Below is Jenkins' stat block.

\begin{monsterbox}{Jenkins Malone}
  \begin{hangingpar}
    \textit{Medium human wizard, lawful evil}
  \end{hangingpar}
	\dndline%
	\basics[
    armorclass = 12,
    hitpoints  = 26,
    speed      = 30 ft
	]
	\dndline%
	\stats[
    DEX = \stat{14},
    CON = \stat{13},
    INT = \stat{18},
    WIS = \stat{14},
    CHA = \stat{8},
	]
	\dndline%
	\details[%
	  languages = {Common},
    challenge = {900 XP}
	]
	\dndline%
	\begin{monsteraction}[Keen Senses]
    Jenkins has advantage on wisdom checks to listen and spot
	\end{monsteraction}

	\monstersection{Actions}

  \begin{monsteraction}[Long Sword]
    +0; \dice{1d8 + 0} slashing damage
  \end{monsteraction}

	\begin{monsteraction}[Cantrips]
    Jenkins knows the \textit{light, mage armor, mage hand, minor illusion, shocking grap,} and \textit{ray of frost} cantrips.
	\end{monsteraction}

  \begin{monsteraction}[Level 1 Spells (4 slots)]
    Jenkins knows \textit{burning hands, comprehend languages, detect magic, disguise self, magic missile,} and \textit{sleep}.
  \end{monsteraction}

  \begin{monsteraction}[Level 2 Spells (3 slots)]
    Jenkins knows \textit{knock, invisibility, scorching ray} and \textit{web}.
  \end{monsteraction}

\end{monsterbox}

\subsection{Death of Rasheed Jules}
Rasheed is the first of the passengers to be killed.  His death is fairly secret, as Jenkins is hoping to start with a low profile.  One he notices other passengers on the train start looking for a killer, he starts to get wilder and more obvious with his murders.

At some point while the players are in the Lounge Car, have Rasheed stand up and leave the room.  He never returns, but is instead hidden in a compartment in the caboose.  This compartment can be found via a DC 13 investigation check.

\begin{paperbox}{Rasheed's Body}
  The body, if found, is covered in a spider's web with some parts cut open around the arms and hands (looking for the tattoos and amethyst ring).  Rasheed's neck has been cut open, with lots of it spilling over the web he's entangled in.
\end{paperbox}

If nobody notices his death after scanning the train for a bit, have Helga (the half-orc Rasheed was talking with) ask if anyone has seen him recently.  This should hopefully spark some interest.  Furthermore, have the luggage car be covered in the remnants of the webs that Jenkins used to trap Rasheed.  Rasheed's death isn't meant to be the first one discovered by the players, but it is meant to add to the feeling of something wrong going on aboard the train.

\subsection{Death of Cassandra Troy}
Cassandra's death is a bit more high profile.  The passengers hear a scream from near the front of the train.  They can enter the coal car to find Cassandra's body, bleeding out on the floor.  Jenkins is also present in the room, but has cast \textit{invisibility} on himself and is thus unable to be seen.

Regardless of if the players rush to check on the scream, Helga will leave to see what's happened.  Furthermore, Charlene (the Warlock from the Passenger car) will enter the lounge, asking about the noise.  Bartholomew will notably \textit{not} be bothered by the scream, because he believes everyone aboard is below him.

\begin{paperbox}{Cassandra's Body}
  Cassandra is covered in the same webs as Rasheed, but her body is left out in the open.  Her throat has also been slit, and she is still dying as the players enter.  If the players can free her quickly, she can offer a clue, but she is beyond saving without serious magical healing.  Near her body, on the floor, is her amethyst ring.
\end{paperbox}

Afterwards, the players will hopefully begin interrogating the different members of the train.  It's important to note that the other passengers should treat the other players suspciously as well.  Jenkins will continue to hide (perhaps the players will think him dead), but, to avoid suspicion, he will wait a bit longer until his next kill.

\subsection{Death of Bartholomew Higgins}
Bartholomew is the next to die.  He is likely the character the players suspect of having something to do with the deaths on the train.  His death will absolve him of guilt at an unfortunately high price.

As Helga wants to keep an eye on our travellers, she will travel with them.  This leaves Bartholemew as the only person left in the lounge.  Again, he does not care about the other passengers and doesn't want to get involved with the investigation; he will remain in the lounge.

This, however, proves to be his downfall, as he will be killed by Jenkins if the players leave him alone for too long.  The players will return to find his body covered in webs, his fingers cut off, with all of his rings scattered onto the floor.

\section{Escalation of Events}
At this point, Jenkins knows that he hasn't killed his mark, but there are too many witnesses to leave onboard the train.  As such, he hopes to pin the blame on a beast that he summons: A Quaggoth Thonot!

\begin{monsterbox}{Quaggoth Thonot}
  \begin{hangingpar}
    \textit{Medium humanoid (quaggoth), chaotic netural}
  \end{hangingpar}
	\dndline%
	\basics[%
  armorclass = {13 (natural armor)},
	hitpoints  = \dice{6d8 + 18},
  speed      = {30 ft., climb 30ft.}
	]
	\dndline%
	\stats[
    STR = \stat{17},
    DEX = \stat{12},
    CON = \stat{16},
    INT = \stat{6},
    WIS = \stat{12},
    CHA = \stat{7},
	]
	\dndline%
	\details[%
    skills = {Athletics +5},
    damageimmunities = {poison},
    conditionimmunities = {poisoned},
    senses = {darkvision 120ft., passive Perception 10},
	  languages = {Undercommon},
    challenge = {3 (700XP)}
	]
	\dndline%

	\begin{monsteraction}[Wounded Fury]
    While it has 10 hit points of fewer, the quaggoth has advantage on attack rolls.  In addition, it deals an extra \dice{2d6} damage to any target it hits with a melee attack.
	\end{monsteraction}

	\monstersection{Actions}
	\begin{monsteraction}[Multiattack]
    The quaggoth makes two claw attacks.
	\end{monsteraction}

	\begin{monsteraction}[Claw]
    \textit{Melee Weapon Attack:} +5 to hit, reach 5ft., one target.
    \textit{Hit:} \dice{1d6 + 3} slashing damage.
	\end{monsteraction}

  \begin{monsteraction}[Innate Spellcasting (Psionics)]
    The quaggoth's innate spellcasting ability is Wisdom (spell save DC 11).  The quaggoth can innately cast the following spells, requiring no components:

    At will: \textit{feather fall, mage hand}

    Once per day: \textit{cure wounds, enlarge/reduce, heat metal, mirror image}
  \end{monsteraction}
\end{monsterbox}

\begin{commentbox}{This Is a Comment Box!}
  A \lstinline!commentbox! is a box for minimal highlighting of text. It lacks the ornamentation of \lstinline!paperbox!, but it can handle being broken over a column.
\end{commentbox}

\subtitlesection{Weapon, +1, +2, or +3}
{Weapon (any), uncommon (+1), rare (+2), or very rare (+3)}

\lipsum[3]

\begin{paperbox}[float=!t]{Behold, the Paperbox!}
  The \lstinline!paperbox! is used as a sidebar. It does not break over columns and is best used with a figure environment to float it to one corner of the page where the surrounding text can then flow around it.
\end{paperbox}

% For more columns, you can say \begin{dndtable}[your options here].
% For instance, if you wanted three columns, you could say
% \begin{dndtable}[XXX]. The usual host of tabular parameters are
% available as well.
\header{Nice table}
\begin{dndtable}
   	\textbf{Table head}  & \textbf{Table head} \\
   	Some value  & Some value \\
   	Some value  & Some value \\
   	Some value  & Some value
\end{dndtable}

% You can optionally not include the background by saying
% begin{monsterboxnobg}
\begin{monsterbox}{Monster Foo}
  \begin{hangingpar}
    \textit{Small metasyntactic variable (goblinoid), neutral evil}
  \end{hangingpar}
	\dndline%
	\basics[%
	armorclass = 12,
	hitpoints  = \dice{3d8 + 3},
	speed      = 50 ft
	]
	\dndline%
	\stats[
    STR = \stat{12}, % This stat command will autocomplete the modifier for you
    DEX = \stat{7}
	]
	\dndline%
	\details[%
	% If you want to use commas in these sections, enclose the
	% description in braces.
	% I'm so sorry.
	languages = {Common Lisp, Erlang},
	]
	\dndline%
	\begin{monsteraction}[Monster-super-powers]
		This Monster has some serious superpowers!
	\end{monsteraction}
	\monstersection{Actions}
	\begin{monsteraction}[Generate text]
		This one can generate tremendous amounts of text! Though only when it wants to.
	\end{monsteraction}

	\begin{monsteraction}[More actions]
    See, here he goes again! Yet more text.
	\end{monsteraction}
\end{monsterbox}

\section{Spells}

\begin{spell}
	{Beautiful Typesetting}
	{4th-level illusion}
	{1 action}
	{5 feet}
	{S, M (ink and parchment, which the spell consumes)}
	{Until dispelled}
	You are able to transform a written message of any length into a beautiful scroll. All creatures within range that can see the scroll must make a wisdom saving throw or be charmed by you until the spell ends.

	While the creature is charmed by you, they cannot take their eyes off the scroll and cannot willingly move away from the scroll. Also, the targets can make a wisdom saving throw at the end of each of their turns. On a success, they are no longer charmed.
\end{spell}

\lipsum[2]

\section{Colors}

This package provides several global color variables to style \lstinline!commentbox!, \lstinline!quotebox!, \lstinline!paperbox!, and \lstinline!dndtable! environments.

\begin{dndtable}[lX]
  \textbf{Color}         & \textbf{Description} \\
  \lstinline!commentboxcolor! & Controls \lstinline!commentbox! background. \\
  \lstinline!paperboxcolor!   & Controls \lstinline!paperbox! background. \\
  \lstinline!quoteboxcolor!   & Controls \lstinline!quotebox! background. \\
  \lstinline!tablecolor!      & Controls background of even \lstinline!dndtable! rows. \\
\end{dndtable}

See Table~\ref{tab:colors} for a list of accent colors that match the core books.

\begin{table*}
  \begin{dndtable}[XX]
    \textbf{Color}                            & \textbf{Description} \\
    \lstinline!PhbLightGreen!                      & Light green used in PHB Part 1 \\
    \lstinline!PhbLightCyan!                       & Light cyan used in PHB Part 2 \\
    \lstinline!PhbMauve!                           & Pale purple used in PHB Part 3 \\
    \lstinline!PhbTan!                             & Light brown used in PHB appendix \\
    \lstinline!DmgLavender!                        & Pale purple used in DMG Part 1 \\
    \lstinline!DmgCoral!                           & Orange-pink used in DMG Part 2 \\
    \lstinline!DmgSlateGray! (\lstinline!DmgSlateGrey!) & Blue-gray used in PHB Part 3 \\
    \lstinline!DmgLilac!                           & Purple-gray used in DMG appendix \\
  \end{dndtable}
  \caption{Colors supported by this package}%
  \label{tab:colors}
\end{table*}

\begin{itemize}
  \item Use \lstinline!\setthemecolor[<color>]! to set \lstinline!themecolor!, \lstinline!commentcolor!, \lstinline!paperboxcolor!, and \lstinline!tablecolor! to a specific color.
  \item Calling \lstinline!\setthemecolor! without an argument sets those colors to the current \lstinline!themecolor!.
  \item \lstinline!commentbox!, \lstinline!dndtable!, \lstinline!paperbox!, and \lstinline!quoteboxcolor! also accept an optional color argument to set the color for a single instance.
\end{itemize}

\subsection{Examples}

\subsubsection{Using \lstinline!themecolor!}

\begin{lstlisting}
\setthemecolor[PhbMauve]

\begin{paperbox}{Example}
  \lipsum[2]
\end{paperbox}

\setthemecolor[PhbLightCyan]

\header{Example}
\begin{dndtable}[cX]
  \textbf{d8} & \textbf{Item} \\
  1           & Small wooden button \\
  2           & Red feather \\
  3           & Human tooth \\
  4           & Vial of green liquid \\
  6           & Tasty biscuit \\
  7           & Broken axe handle \\
  8           & Tarnished silver locket \\
\end{dndtable}
\end{lstlisting}

\begingroup
\setthemecolor[PhbMauve]

\begin{paperbox}{Example}
  \lipsum[2]
\end{paperbox}

\setthemecolor[PhbLightCyan]

\header{Example}
\begin{dndtable}[cX]
  \textbf{d8} & \textbf{Item} \\
  1           & Small wooden button \\
  2           & Red feather \\
  3           & Human tooth \\
  4           & Vial of green liquid \\
  6           & Tasty biscuit \\
  7           & Broken axe handle \\
  8           & Tarnished silver locket \\
\end{dndtable}
\endgroup

\subsubsection{Using element color arguments}

\begin{lstlisting}
\begin{dndtable}[cX][DmgCoral]
  \textbf{d8} & \textbf{Item} \\
  1           & Small wooden button \\
  2           & Red feather \\
  3           & Human tooth \\
  4           & Vial of green liquid \\
  6           & Tasty biscuit \\
  7           & Broken axe handle \\
  8           & Tarnished silver locket \\
\end{dndtable}
\end{lstlisting}

\begin{dndtable}[cX][DmgCoral]
  \textbf{d8} & \textbf{Item} \\
  1           & Small wooden button \\
  2           & Red feather \\
  3           & Human tooth \\
  4           & Vial of green liquid \\
  6           & Tasty biscuit \\
  7           & Broken axe handle \\
  8           & Tarnished silver locket \\
\end{dndtable}

% End document
\end{document}
